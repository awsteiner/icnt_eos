\documentclass{revtex4}
\begin{document}
\section*{Equation of State and the Physics of Dense Matter Summary}
\begin{center}
  With contributions from: {\it{Christian Drischler (Darmstadt),
      Francesca Gulminelli (Caen),
      Jeremy Holt (Texas A\&M), Charles Horowitz (IU),
      Morten Horth-Jensen (MSU),
      Francesco Pederiva (Trento), Jorge Piekarewicz (FSU),
      Madappa Prakash (Ohio U.), Luke Roberts (MSU),
      Andre de Silva Schneider (Caltech),
      Andrew W. Steiner (UTK/ORNL)}, Ingo Tews (INT),
    Corbinian Wellenhofer (TU Munich)}
\end{center}

Heavy-ion collisions, in concert with other experimental probes of
nuclear structure, promise to play a critical role in determining
the equation of state and constraining the physics of dense matter.
These experimental constraints are essential for providing the correct
microphysics input to describe neutron star phenomena, core-collapse
supernovae, and neutron star mergers.

\subsection{Near and Below the Nuclear Saturation Density}

Nucleon-nucleon interactions constructed from chiral effective field
theory have been utilized in different many-body approaches to compute
the EOS of nucleonic matter at both zero and finite temperatures.
Historically, it has been difficult to obtain a good description of
both nuclear matter and neutron matter at saturation, but some
recent calculations have been able to achieve this. As emphasized
in the workshop, the principal origin of uncertainty in the EOS now
lies not in the approach to the many-body problem but in the
uncertainty in the interaction itself. 

Describing heterogeneous matter is much more difficult.

\subsection{Above the Nuclear Saturation Density}

Neutron star mass and radius observations have begun to provide
information on the pressure of cold neutron-rich matter. Constraints
on the pressure, however, do not necessarily translate to information
on the composition of neutron star cores. On the question of hyperons,
quantum Monte Carlo results have shown clearly that the threshold
density for the appearance of the $\Lambda$ hyperon vary from just
above saturation to very large densities depending on the repulsion
from the unknown $\Lambda-n-n$ force.

Chiral interactions and phenomenological models like Skyrme and MDI
interactions cannot be easily applied to high-density matter. However,
hot and/or isospin-symmetric matter is difficult for theory and not
yet constrained by astronomical observations. Thus heavy-ion
collisions (through observables like pion ratios) are important in
constraining dense matter.

\subsection{Action Items}

Paragraph regarding JINA-CEE EOS working group.


\end{document}
